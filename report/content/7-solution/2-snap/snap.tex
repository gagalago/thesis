\section{SNAP!}
\label{solution SNAP}
Comme expliqué précédemment, nous sommes partis d'un projet existant, SNAP BYOB, pour l'environnement de programmation. Ce projet a pour but de fournir une interface et un environnement supportant la programmation par bloc. Ce projet ne s'inscrit pas dans le cadre d'un apprentissage scolaire ou guidé. Il a donc fallu adapter le projet pour une utilisation plus scolaire. Nous allons expliquer dans cette partie les différentes adaptations que nous avons apportées au projet d'origine et pourquoi elles sont nécessaire pour remplir nos objectifs.\\

Dans les adaptations opérées, nous retiendrons : une simplification de l'interface, des fonctionnalités supplémentaires telles que la sauvegarde sur les serveurs du projet courant, une différenciation de rôle pour le professeur par rapport au public cible, une amélioration de la traduction en français, le masquage des scripts.

\subsection{L'interface}
\label{interface}
Comme expliqué précédemment le projet original n'avait pas une vocation didactique de groupe et avait un publique cible plus large que le notre. Dans cette idée, il y avait des menus pour des fonctionnalités de gestion de l'ordonnanceur et autres. Il est évident que ces fonctionnalités ne sont pas utiles pour notre public. De plus vu l'âge de notre public toutes distractions qui peuvent être évitées améliorent sensiblement la productivité de ces derniers.

Dans cette optique, nous avons opéré un nettoyage en profondeur de l'interface dans l'optique de laisser uniquement les menus utile. Toute fois comme il sera discuté dans les rôles il était intéressant de ne pas simplement les effacer, mais bien de les masquer.

\subsection{Les rôle}
\label{role}
Dans l'optique de fournir une interface épuré pour les étudiants comme expliqué dans la section \ref{interface}. Il était nécessaire d'enlever des parties de l'interface non pertinente. Toutes foi, ces options inutiles aux étudiant peuvent l'être pour les professeurs ou dans l'optique d'avoir une missions "monde ouvert". Il est donc utile de ne pas dédoubler les interfaces mais bien d'avoir une interface modulaire suivant son utilisation. \\

Pour différencier les utilisations de l'interface, nous avons ajouté la notion de rôles. Nous avons défini deux rôle: étudiant et professeur. Grâce à cela, quand les étudiant ouvre un projet, SNAP! est lancé avec le rôle étudiant ce qui permet de ne pas affiché les options superflues. Quand c'est un professeur qui souhaite modifier une mission ou en crée une nouvelle, RSNAP lance SNAP avec le rôle professeur ce qui permet d'avoir accès au fonction avancée de SNAP.\\

Ces rôles permettent également la gestion du masquage de script. En effet il est intéressant qu'un professeur puisse masquer des blocs, mais cela ne sert à rien si les étudiants peuvent les ré-afficher.

\subsection{Masquer les scripts}

\subsection{Fonctionnalités supplémentaires}
Comme nous avons interfacé l'application de programmation avec un site web, nous avons du reimplementer certaine fonctionnalité d'import-export.\\

Nous avions besoin de pouvoir passer le projet à l'ouverture de l'application. Pour cela, la majorité des fonctions était déjà présente. La technique utiliser était de passer l'XML contenant le projet dans la barre d'adresse du navigateur. Quelques adaptations on permis de masquer le passage du projet a l'utilisateur.\\

Dans les fonctions d'export, il y avait ici aussi des solutions existantes, mais encore une fois, n'étant pas la priorité des personnes maintenant le projet, ces fonctions étaient peu pratiques. Lors d'un export de projet, le fichier XML généré était généré dans une page HTML. Nous avons repris ces fonctions afin de créer le fichier HTML dans les temps ensuite de l'envoyer sur le serveur à l'aide d'une requête \texttt{push}

tous les changements de l'interface

bouton d'aide

rôles

export import

Traduction ???

aides

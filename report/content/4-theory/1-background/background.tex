\section{Prérequis}
C'est dans cette partie que les connaissances nécessaires à la bonne compréhension de ce travail vont être développées. Commençons par la partie portant sur SNAP BYOB, cette application est implémentée en JavaScript. Par ce fait, de bonnes connaissances dans ce langage de programmation seront un atout pour comprendre les apports et modifications apportées à la version de départ. Des connaissances en Json seront également un plus pour toutes les fonctions d'importation et d'exportation.\\

La partie site web, elle est implémentée en RubyOnRails qui est le framework le plus utilisé pour la création de site web en ruby. Ce framework est un atout, car il permet de créer facilement des sites sur une architecture MVC. Rail se base sur des conventions à la place de se baser sur des configurations, ce point permet d'alléger considérablement le nombre de lignes nécessaire pour la création d'un site web. Ce framework étant basé sur Ruby permet l'usage des gems ruby et donc d'avoir facilement des fonctionnalités supplémentaires.\\

Pour les interactions ou la création de contenu pour les jeunes, des notions de pédagogie sont également nécessaires. Les programmes d'apprentissages existants permettent de situer les concepts considérés comme acquis suivant l'année d'étude du jeune. Toute la partie communication avec les jeunes requiert également des notions quant au langage utilisé. Elles s'avèrent importantes dans des choix de design d'interface. En effet, suivant leur âge les jeunes ne seront pas stimulés par les mêmes images ou couleurs.\\

Nous avons repris un projet libre existant qui est SNAP BYOB et qui est un environnement de programmation par blocs. Comme dit précédemment, cet environnement est écrit en JavaScript. Nous allons ici présenter le projet et ses origines. Le responsable du projet est jimoegning, c'est lui qui centralise et gère toute l'évolution du projet. Nous devons énormément à ce projet, car sans lui nous n'aurions pas pu aller aussi loin dans le développement de l'application. SNAP BYOB est un projet qui a commencé en 2011 et qui est maintenant à la version 4, qui est celle que nous utilisons.

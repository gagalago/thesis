\subsection{L'état de la programmation dans le monde}
Cette partie fait un tour d'horizon de différent pays qui enseignent la programmation aux jeunes. Spécialement sur le positionnement politique que ces pays ont adopté par rapport à notre problématique. Cette partie ne porte donc plus sur une méthode ou une organisation mais bien sur le contexte et la manière d'intégré l'apprentissage de la programmation dans chaque pays.
\subsubsection{Angleterre}
L'apprentissage de l'informatique en Angleterre\footnote{\url{https://www.gov.uk/government/collections/statutory-guidance-schools\#national-curriculum-from-september-2014}} n'est pas nouveau. Cependant, longtemps cet apprentissage était centré sur les technologies de l'information et de la communication (TIC). En 2010 une étude a été commandée à la Royal Society pour évaluer cet apprentissage. Un an plus tard leur rapport a révélé que l'enseignement tel que dispensé jusque là, n'était ni efficace ni en adéquation avec l'évolution de l'informatique dans notre société. La Royal Society suggère de changer les matières abordées en informatique. En effet précédemment c'était les TIC qui étaient enseigné alors que l'apprentissage de la programmation serai plus bénéfique et adapter pour les enfants. Sur base de ce rapport les programmes de cours ont été adaptés.

\subsubsection{France}
En France\footnote{\url{http://fr.wikipedia.org/wiki/Informatique\_et\_sciences\_du\_num\%C3\%A9rique}
\url{http://fr.wikipedia.org/wiki/Baccalaur\%C3\%A9at\_scientifique}} depuis deux an l'informatique fait partie intégrante du programme du baccalauréat de type S. Une des matière dispensée est "Informatique et sciences du numérique". Cette matière se subdivise en quatre sous parties qui sont : représentation de l'information, algorithmique, langage et programmation, architectures matérielles. Cette approche est donc également basé sur l'apprentissage de la programmation plutôt que sur les TIC.

\subsubsection{Nouvelle Zélande}
Ce pays à adopté récemment les sciences informatique dans sont programme d'étude. Les cours sont dispenser à partir de 15 ans. Les cours dispensé concerne l'apprentissage de la programmation et des concepts informatique en général. Nous avons un choix un peu différent dans ce pays sur l'âge du début de l'apprentissage. Beaucoup de pays commence plus jeune et introduise des concepts basiques. La Nouvelle Zélande s'inscrit dans une logique plus similaire à la France mais ne restreint pas l'informatique aux options scientifiques.

\subsubsection{Corée du Sud}
Enseigne l'informatique depuis longtemps et à tous les niveaux de l'enseignement. La culture numérique dans ces pays y est fort différent que par chez nous. Par exemple une carrière dans le gaming y est tout à fait normal. L'informatique est vraiment omni-présent dans ces cultures, il est donc normal que son apprentissage commence en primaire.

\subsubsection{Grèce}
L'apprentissage des sciences informatiques prend une place important dans les programme grec. Des 6 ans les enfants sont confronté à l'informatique à l'école. A cette age c'est plus de la maîtrise de l'outil qu'il apprennent. Dès 10 ans, leurs cours d'informatique prend une tournure plus algorithmique et donc plus proche de la computer sciences.

\section{Travail associé}
Nous allons ici présenté un aperçu des initiatives similaires à Rsnap. En commençant par un aperçu des principaux acteur du secteur, suivit d'un aperçu des politique de plusieurs pays, qui prône l'apprentissage de l'informatique, à travers le monde. Nous clôturerons ce chapitre par une liste des concepts qui définisse une initiative similaire à celle de ce travail.
\subsection{Code.org}
Code.org\footnote{\url{http://code.org/about}} est une organisation sans but lucratif des USA qui à pour objectifs :
\begin{itemize}
  \item Apporter l'informatique dans toutes les classes de secondaire aux USA;
  \item Démontrer le succès de l'utilisation de cours en ligne dans l'enseignement public;
  \item Ajouter l'informatique dans les bases des programmes de sciences/math des 50 états;
  \item Employé la connaissance technique collective pour améliorer l'apprentissage de l'informatique dans le monde;
  \item Augmenter la représentation féminine et des personnes de couleurs dans informatique.
\end{itemize}

Pour ce faire, il fournisse une plat-forme\footnote{\url{http://code.org/educate/20hr}} web qui permet aux professeurs de suivre leur élèves grâce à un système de classe.

Toutes les ressources sont gratuites et librement utilisable\footnote{\url{http://code.org/faq}}. Leur programme d'apprentissage se base sur Blockly (voir \ref{blockly}).
Les ressources sont conçues pour que les professeurs comme les étudiants puissent commencer le cours sans connaître l'informatique (un assistance est proposé au professeur gratuitement si nécessaire).

Le site propose aux professeurs de se faire récompenser si ils arrivent à finir les 27 missions proposée à  minimum 15 étudiants. Dans ce cas ils gagnent $750\$$, si ils ont au moins 7 filles dans le groupe ils peuvent prétendre a $250\$$ supplémentaire.

\subsubsection{Déroulement des lecons}
Il propose des session de une heure de travail/jeu/apprentissage. Chaque unité d'une heure est découper en petite mission (ex:5-20) les missions sont très courte et apporte un concept de programmation. Avant l'introduction de chaque concept une petite vidéo est faire pour expliquer le concept introduit et des exemples de ce que la programmation permet de réaliser avec ce dernier.

Ils proposent de faire travailler les étudiants par pair\footnote{pair programming \url{http://en.wikipedia.org/wiki/Pair\_programming}}. Ceci permet d'avoir moins de questions pour le professeur et de mieux s'approprier la matière. Le travail par pair permet également de casser l'image du "geek" en montrant que la programmation est une sciences sociale et collaborative. Sans oublier qu'avec deux enfants par station, moins d'ordinateurs seront nécessaire.

Le site explique également que pour faire participer tout les élèves, on avoir confiance en leur compétence : permettre aux premiers groupes d'aider les derniers.

Pour résoudre un problème, ils recommandent de proposer au étudiants de d'abord demander à 3 de leur camarades avant de poser la question au professeur. Le prof ne devant pas être compétent, il doit juste pouvoir réfléchir avec les élèves de quel est le problème, cela permet aussi d'évité les questions de distraction ou de manque de compréhension.

Pour chaque petite mission il y a un test automatisé qui dit si la mission est réussie ou non. Si la mission est réussie, le programme passe à la mission suivante. Il y a également un compteur de blocs dans les première mission. Ce compteur permet de voir combien de bloc sont nécessaire pour réaliser la mission de manière optimale.

\subsection{CoderDojo}
CoderDojo\footnote{\url{http://coderdojo.com/about}} est un réseau open source de clubs de programmation dans le sens le plus large du terme. Tous les dojos sont donc autonome.  Dans ceux-ci, des enfant de 5 à 17 ans apprennent la programmation (site web, application, jeux...). la seul règle est  “Above All: Be Cool“ qui peut être mise en pratique simplement en créant des espace d'échanges de savoir amicale et sociable.

CoderDojo à été crée par James Whelton un irlandais de 18 ans et Bill Liao un entrepreneur australien à Cork. James a eu des demandes de jeunes enfants pour avoir des cours de programmation après qu'il eut hacké l'ipod nano. Pas mal de gens de Dublin vinrent à ses cours et donc un nouveau Dojo à été crée à Dublin et puis cela s'est étendu à tout le globe.

\subsection{Code Club}

Code Club\footnote{\url{https://www.codeclub.org.uk/about}} est un réseau de club national mené par des bénévole en dehors des heures de cours. Ces activités s'adressent à des enfants entre 9 et 11 ans.

Ils créent donc le matériel pour permettre à des bénévole de donner des cours parascolaire d'environ une heure semaine. Il propose dans l'ordre d'utiliser scratch, html/css et python. ils aimeraient que les 21 000 écoles primaires anglaises ai leur club.

Leur philosophie est de d'abord l'amusement, la créativité et l'exploration avant l'apprentissage des concept de programmation.

\section{Langages de programmation}
Nous allons ici expliquer les principaux langages de programmation 
\subsection{Blockly}
\label{blockly}
Blocky\footnote{\url{https://code.google.com/p/blockly/}} est un éditeur de programmation graphique basé sur des technologie du web. 

Blocky est influencé\footnote{\url{https://code.google.com/p/blockly/wiki/Alternatives}} par "App Inventor" qui est influencé par "Scratch". Ce dernier est lui influencé par "StarLogo".

Il a comme particularité:
\begin{enumerate}
\item De s'exécuter dans un navigateur;
\item D'exporter du code source en JavaScript, dart, etc..;
\item D'être open source;
\item D'être haut-niveau.
\end{enumerate}

Il n'est pas directement une plat-forme d'éducation dans le sens ou suivant les blocs implémenté, il sera utilisé pour l'éducation, le business, des jeux, etc..


Lors de la conception du language de blocky il devait avoir certaines propriétés. Les trois première sont pour augmenter la compréhension des néophyte, les autres portait sur des facilité souhaiter du langage. Les propriétés décidées lors de la conception du langage sont:\footnote{\url{https://code.google.com/p/blockly/wiki/Language}}:

\begin{itemize}
  \item Des index de liste commençant à 1;
  \item Des nom de variables non sensible à la casse;
  \item Pas de scoope de variable, toutes les variables sont globale;
  \item La possibilité de pouvoir faire un export en JavaScript;
  \item Un code natif généré proche de celui des blocs.
\end{itemize}

\subsection{SNAP}
SNAP est un langage de programmatation de glisser-déposer de blocs. c'est une ré-implémentation et une extension du langage Scratch du MIT. C'est un langage qui a été pensé et conçu pour être orienté web, ce langage est donc implémenté en JavaScript.\\

Ce langage est né en 2011 et a été créer par Jens Mönig. Il se distingue de son père Scratch par l'ajout de:
\begin{enumerate}
\item Des fonctions et des procédures de première classe;
\item Des listes de première classe;
\item Des sprite de première classe.
\end{enumerate}

\subsection{Scratch 1}
Scratch est un langage de programmation développé par le MIT pour apprendre aux enfants la programmation. C'est interface qui permet de faire des scripte facilement grâce à sa programmation par bloc.\\

Ce langage à été pensé pour être un outils créatif pour réaliser des histoires, des jeux, des simulations, de l'art, etc.. Ce langage a par exemple son propre éditeurs d'image et de sons. Un autre but est que ce langage doit être simple à utiliser et à apprendre. Il a en effet été conçu pour des enfants n'ayant aucune connaissance préalable en programmation.\\

L'interface de Scratch se divise en plusieurs grandes parties:
\begin{enumerate}
\item Sur la gauche il y a la zone de dessin dans laquelle s'anime les composants graphique des scriptes;
\item Au milieu il y a une liste des blocs disponible trier par catégorie.
\item Sur la droite il y a la zone de scripte qui contient tout les scripte lié au sprit sélectionné.
\end{enumerate}

\subsection{Python}


\section{L'état de la programmation dans le monde}
Nous allons ici faire un tour d'horizon de différent pays ou non qui enseignent la programmation aux jeunes. 
\subsection{Angleterre}

L'apprentissage de l'informatique en Angleterre\footnote{\url{https://www.gov.uk/government/collections/statutory-guidance-schools\#national-curriculum-from-september-2014}} n'est pas nouveau. Pendant longtemps cet apprentissage était centré sur les technologies de l'information et de la communication (TIC). En 2010 un étude a été commandée à la Royal Society pour évaluer cet apprentissage. Un an plus tard leur rapport a révélé que l'enseignement tel que dispensé jusque là, n'était ni efficace ni en adéquation avec l'évolution de l'informatique dans notre société. La Royal Society suggère de changer les matières abordées en informatique. En effet précédemment ce sont les TIC qui étaient prescrites. L'apprentissage de la programmation serai plus bénéfique et adapter pour les enfants. Sur base de ce rapport les programmes de cours ont été adaptés.


\subsection{France}
En France\footnote{\url{http://fr.wikipedia.org/wiki/Informatique\_et\_sciences\_du\_num\%C3\%A9rique}
\url{http://fr.wikipedia.org/wiki/Baccalaur\%C3\%A9at\_scientifique}} depuis deux an l'informatique fait partie intégrante du programme du baccalauréat de type S. Une des matière dispensée est "Informatique et sciences du numérique". Cette matière se subdivise en quatre sous parties qui sont : représentation de l'information, algorithmique, langage et programmation, architectures matérielles. Cette approche est donc également basé sur l'apprentissage de la programmation plutôt que sur les TIC.

\subsection{Nouvelle Zélande}
Ce pays à adopté récemment les sciences informatique dans sont programme d'étude. Les cours sont dispenser à partir de 15 ans. Les cours dispensé concerne l'apprentissage de la programmation et des concepts informatique en général. Nous avons un choix un peu différent dans ce pays sur l'âge du début de l'apprentissage. Beaucoup de pays commence plus jeune et introduise des concepts basiques. La Nouvelle Zélande s'inscrit dans une logique plus similaire à la France mais ne restreint pas l'informatique aux options scientifiques.

\subsection{Corée du Sud}
Enseigne l'informatique depuis longtemps et à tous les niveaux de l'enseignement. La culture numérique dans ces pays y est fort différent que par chez nous. Par exemple une carrière dans le gaming y est tout à fait normal. L'informatique est vraiment omni-présent dans ces cultures, il est donc normal que son apprentissage commence en primaire.

\subsection{Grèce}
L'apprentissage des sciences informatiques prend une place important dans les programme grec. Des 6 ans les enfants sont confronté à l'informatique à l'école. A cette age c'est plus de la maîtrise de l'outil qu'il apprennent. Dès 10 ans, leurs cours d'informatique prend une tournure plus algorithmique et donc plus proche de la computer sciences.

\section{Similitudes / Différences}
Maintenant que nous avons vu comment fonctionnaient différentes structures à travers le monde, essayons de synthétiser les différentes et similitudes de l'apprentissage de la programmation.
\subsection{pour qui ?}
L'apprentissage de la programmation aux enfants ne se fait pas de la même manière avec tous ceux-ci. Les différentes organisations auront donc soit un public visé restreint soit différents programmes pour les différents groupes d'enfants. 

\subsubsection{Age}
Les différences peuvent être simplement du à des différences d'ages. La méthode d'apprentissages doit être différents suivant l'age. Les enfants n'ont pas toujours la maturité d'apprendre des concepts parfois fort abstrait (boucle, parallélisation...). En effet pour des élèves de maternel par exemple un tableau interactif sera un grand plus car ils n'ont pas encore toute la dextérité souhaitée sur la souris. Et au contraire des enfants plus grands seront plus autonomes et pourront travailler par paire voir seul.

\subsubsection{Origine et genre}
Une autre différence est l'origine des enfants, certaines organisations, tel Code.org, mettent l'accent sur la disponibilité pour tous et donc notamment pour les filles qui sont sous représentée en informatique. Tout comme il est important de ne pas viser que les personnes qui peuvent acquérir un ordinateur à la maison ou se payer un stage. Dans ce cas pour toucher un maximum de personne l'école reste l'endroit le plus adéquat

\subsection{Commment ?}
La principale différences entre toutes les organisations va être sur la manière d'aborder l'apprentissage. Les différences peuvent être de plusieurs ordres, tel que : les outils utilisés, les concepts abordés et l'environnement de travail.

\subsubsection{Outils} 
Un aspect très pratique qui différencies l'apprentissage est les outils utilisés. Ici encore différentes écoles s'affrontent :
\begin{description}
  \item[Langage réel] Les enfants doivent apprendre ce qu'est la programmation et donc ce qu'est du code source. De plus, avec un langage déjà couramment utilisé, plus de choses sont possibles.
  \item[Langage visuel] Dans la programmation, le plus important est la logique. Donc pourquoi ne pas avoir un langage visuel par blocs pour simplifier la syntaxe et se rapprocher un peu de ce qui se fait généralement avec des diagrammes.
  \item[Langage web] Dans le monde d'aujourd'hui, le web est partout, plus que tout le reste. l'accent doit donc être porté sur l'apprentissage de ce que les enfants voient au quotidien.
\end{description}

\subsubsection{Concepts} 
L'ordre dans laquelle les concepts de programmation sont abordés peut se différencier sur plusieurs points.Tout cours se divise en deux morceaux : la théorie et la pratique. Mais il existe de multiples manières d'assembler les deux dans un cours unique.
\begin{itemize}
  \item Certains commencent par donner de la théorie et demandent ensuite de l'appliquer ;
  \item D'autres proposent des exercices avant d'expliquer ce que les enfants ont utilisé de manière intuitive ;
  \item D'autres encore découpent la matière en tout petit sous ensemble avant d'appliquer la première ou la seconde technique.\\
  \end{itemize}


Ces différentes façons d'aborder l'apprentissage mènent à différentes façon de concevoir les missions. Celles-ci peuvent être très dense ou au contraire un succession de mission simple. Les missions dense sont souvent constituées de nombreux concepts de programmation (boucle, condition, fonction...) pour avoir directement un objectif concret alors que les missions simple essayent de bien séparé l'apprentissage de ceux-ci pour fournir une étude la plus progressive possible.\\

Dans certains pays, la programmation est associé au cours de sciences. Elle deviendra donc souvent un outil scientifique et perdra du même coup un peu de son coté ludique. Néanmoins, grâce à cela un plus grand nombre d'enfants pourront toucher à la programmation.\\

Dans tout les cas, il faudra trouver une manière ludique d'apprendre la programmation aux enfants. En effet, les enfants ont besoin de s'approprier la matière. Quoi de mieux que un jeux pour les garder attentif à ce qu'ils réalisent.

\subsubsection{Environement de travail}
L'organisation d'une leçon peut varier dans la gestion du groupe :
\begin{description}
  \item[En groupe] L'apprentissage de groupe est souvent réalisé avec un tableau interactif ou un projecteur \footnote{\url{http://scratched.media.mit.edu/resources/scratch-\%C3\%A0-la-maternelle}} ;
  \item[Par binôme] La programmation en binôme est la plus répandue dans les cours de programmation. Faire travailler les enfants par deux, les obliges à interagir entre eux et à essayer de résoudre eux même les problèmes qu'ils rencontrent. cela permet aussi de diminuer le nombre d'ordinateurs utilisés ;
  \item[Individuel] Cette technique est très peu utilisé car elle demande beaucoup de moyen matériel et personnel ;
  \item[A domicile] De nombreuses plat-formes proposent des cours en lignes qui peuvent être suivi au rythme de chacun. Néanmoins, pour avoir de l'aide, il faudra se tourner vers une personne compétente ou vers les forums. Ces formations sont souvent plus théorique ou demande plus de lecture, ce qui empêche les jeunes enfants de les suivre.
\end{description}

\section{Par qui ?}
Les cours de programmation sont fournit par des organisations. Celles-ci fournissent des programmes de cours à des enseignants. La variété de ces différents aspects va être abordé dans la suite.

\subsubsection{Type d'organisation}
Différentes organisations proposent des cours de programmations.On peut déjà différencier le fait que l'informatique fasse partie du programme officiel ou pas ou plus simplement que l'apprentissage se déroule à durant les cours ou en parascolaire.

~\\

Dans le premier cas, les étudiants ne viennent pas tous par choix contrairement au second. Ceci peut entraîner des blocage et des réticences à participer correctement à l'activité, comme cela peut arriver pour tout les autres cours dans une école.

Pour les cours donnés en parascolaire, aussi bien de petites structures très locales, se partagent le terrain avec de grandes organisations à visée internationale. Les problèmes typique de ce genre d'organisation est le coût des activités et les horaires.

\subsubsection{Enseignants}
La plupart des cours sont donnés par des informaticiens ou au minimum des personnes formées à l'informatique. C'est bien entendu la manière la plus logique de travailler pour fournir un enseignement de qualité. Mais comme le nombre d'informaticien est réduit, certaines organisations propose des cours clés en mains pour des enseignants sans connaissances particulières en programmation. Ces professeurs apprendront en même temps que leur étudiants et seront la uniquement pour les aider à réfléchir.\\

Il n'y a pas de tendance globales sur le type de rémunération des enseignants. Ils recevront parfois des prime en fonction de la réussite de leur classe, d'autre seront payé directement par les étudiants ou par l'état, d'autres encore seront bénévoles.

\subsubsection{Création des cours}
Suivant l'organisation, les cours seront plus ou moins figé. Certaines organisations permettent à leur communauté de modifier, créer des cours tandis que d'autres ont des équipes dédié à cette tache.


\section{Positionnement de Rsnap}
Cette partie a pour but de positionner Rsnap au travers des concepts mis en lumière par le chapitre précédent.

\paragraph{Age, origine et genre} Rsnap est une application qui vise un public entre 10 et 14 ans. Les missions qui ont été développés dans le cadre de ce travail, cible cette tranche d'âge particulièrement. Cette tranche sera nuancée lors de l'analyse des résultats de l'expérimentation sur les différentes tranche d'âge (voir \ref{trancheage}). %TODO faire la référence quand on écrira les tranche d'age
Toute fois l'application permettant la création de mission de manière autonome, il n'est pas exclut de créer des missions pour d'autre tranche d'âge. La créativité des utilisateurs est la seul limite formelle.\\

Quant au origine et au genre, aucune discrimination dans quelque sens que ce soit n'a été faite lors de la réalisation de ce travail. Tant pour attirer un publique peu représenté que pour l'exclure. Hormis le fait que l'application à été pensé pour un public francophone pour des raison évidente, c'est une problématique qui n'a tout simplement pas été tenue en compte. Comme l'idée est de proposer l'application dans les écoles de la comunauté francaise, cette problématique est sans objet.

\paragraph{Outils, concepts et environnement de travail} Rsnap est basé sur le langage Snap! BYOB qui est un langage visuel. Ce choix s'explique par les objectifs qui sous-tendent ce travail, à savoir l'apprentissage de la programmation dans le but d'amélioré l'esprit logique des jeunes. L'accent étant mis sur la logique sous-jacente et, dans un premier temps, pas sur l'apprentissage d'un langage de programmation. Un but de l'application Rsnap étant également de susciter des vocations, ces vocation se dirigeront d'elle même vers un vrai langage de programmation pour lequel il existe déjà beaucoup de cours. 

Le fait de dissocier l'apprentissage de la logique et d'un langage de programmation était également importante lors de ce travail. En effet il faut diviser pour mieux régner et vouloir tout faire en même temps ne convient pas à tous. En se concentrant sur la logique, cela assure une plus grande confiance dans l'acquisition de la matière.

Un de nos objectif était que les personne qui guide l'activité n'ait pas besoin de connaissance particulière en programmation. De par cet objectif, un vrai langage de programmation a été écarté car ils auraient du au minimum avoir connaissance de sa syntaxe.\\

La manière dont les concepts sont abordés dans Rsnap se fait par la procédure suivante:
\begin{itemize}
	\item Une vidéo d'introduction à la mission ;
	\item Un texte descriptif de la mission qui reprend les concepts théoriques qui seront introduit dans la mission ;
	\item Une fois dans le programme les jeunes n'ont plus l'explication de manière explicite, cela les incite à travailler avec leur intuition tout en pouvant si besoin récupéré la description du point 2 ;
	\item Une page d'aide est disponible pour chaque bloc dans le menu contextuel.
\end{itemize}

En plus de cela, les missions implémenté dans ce travail sont des petites missions introduisant un à deux grands concepts maximum par mission. Ces petites missions ont pour but d'être assemble pour faire un programme final plus grand. Plus d'information à propos du découpage et du contenu des missions est disponible dans la section \ref{missions}.\\

L'environnement de ce travail a été déterminer en grande partie par le milieu d'utilisation de l'application, à savoir les écoles. L'objectif d'indépendance des jeune par rapport au référent a orienté l'activité vers la formation de binôme. Toute fois comme le groupe animé est une classe une introduction collective est possible et nécessaire. Ce travail étant basé sur une plat-forme web, et l'accès au site n'étant pas limité, si les jeunes souhaite utiliser le site en dehors des heure de cours, cela est également possible.
Nous avons donc un travail lors des séances collective qui s'effectue par binôme avec une possibilité de travail en dehors des séances prévue à cet effet.

\paragraph{Type d'organisation, enseignants, création des cours}Le type d'organisation à l'heure actuel est sans objet puisque aucune organisation officielle n'a été créée.\\

A propos des enseignants, les activités ont été dispensé par les auteurs de ce travail. Toute fois le projet à été créé dans le but que la personne qui dispense l'activité n'ait pas besoin de connaissance spécifique en programmation. Le simple fait de réaliser les projets avant les jeunes devrai être suffisant pour acquérir la logique nécessaire pour la transmettre.\\

La création des cours est un point sur lequel Rsnap se distingue de beaucoup d'autre par le fait qu'une partie des missions existent déjà et fond partie de ce travail. Mais les professeurs peuvent également créer des missions et les partager avec le reste de la communauté. Tout comme ils peuvent également reprendre des missions existante et les améliorer ou les adapter. %TODO voir si il faut parler du fait qu'on chapeautera le tout.
\section{Concepts différenciateurs}
\label{concepts}
Maintenant à travers le fonctionnement des  différentes structures à travers le monde vu au point précédent, ce chapitre va mettre en lumière les différences et similitudes des méthodes d'apprentissage de la programmation. Ceci grâce à une série de critères tels que le public cible, le lieu, les moyens et les personnes.
\subsection{Pour qui ?}
L'apprentissage de la programmation aux enfants ne se fait pas de la même manière avec les différentes tranches d'âge. Les différentes organisations auront donc soit un public visé restreint soit différents programmes pour les différents groupes d'enfants. 

\subsubsection{Âge}
La méthode d'apprentissages doit être différente suivant l'âge. Les enfants n'ont pas toujours la maturité d'apprendre des concepts parfois fort abstraits (boucle, parallélisation...). En effet pour des élèves de maternel par exemple un tableau interactif sera un grand plus, car ils n'ont pas encore toute la dextérité souhaitée sur la souris. Et au contraire des enfants plus grands seront plus autonomes et pourront travailler par paire voir seul.

\subsubsection{Origine et genre}
Une autre différence est l'origine des enfants, certaines organisations, telle Code.org, mettent l'accent sur la disponibilité pour tous et donc notamment pour les filles qui sont sous représentés en informatique. Tout comme il est important de ne pas viser que les personnes qui peuvent acquérir un ordinateur à la maison ou se payer un stage. Dans ce cas, pour toucher un maximum de personne, l'école reste l'endroit le plus adéquat.

\subsection{Comment ?}
La principale différence entre toutes les organisations va être sur la manière d'aborder l'apprentissage. Les différences peuvent être de plusieurs ordres, tels que : les outils utilisés, les concepts abordés et l'environnement de travail.

\subsubsection{Outils} 
Un aspect très pratique qui différencie l'apprentissage est les outils utilisés. Ici encore différentes écoles s'affrontent :
\begin{description}
  \item[Langage réel] Les enfants doivent apprendre ce qu'est la programmation et donc ce qu'est du code source. De plus, avec un langage déjà couramment utilisé, plus de choses sont possibles.
  \item[Langage visuel] Dans la programmation, le plus important est la logique. Donc, pourquoi ne pas avoir un langage visuel par blocs pour simplifier la syntaxe et se rapprocher un peu de ce qui se fait généralement avec des diagrammes.
  \item[Langage web] Dans le monde d'aujourd'hui, le web est partout, plus que tout le reste. L’accent doit donc être porté sur l'apprentissage de ce que les enfants voient au quotidien.
\end{description}

\subsubsection{Concepts} 
L'ordre dans laquelle les concepts de programmation sont abordés peut se différencier sur plusieurs points. Tous les cours se divisent en deux morceaux : la théorie et la pratique. Mais il existe de multiples manières d'assembler les deux dans un cours unique.
\begin{itemize}
  \item Certains commencent par donner de la théorie et demandent ensuite de l'appliquer ;
  \item D'autres proposent des exercices avant d'expliquer ce que les enfants ont utilisé de manière intuitive ;
  \item D'autres encore découpent la matière en tout petit sous ensemble avant d'appliquer la première ou la seconde technique.\\
  \end{itemize}


Ces différentes façons d'aborder l'apprentissage mènent à différentes façons de concevoir les missions. Celles-ci peuvent être très dense ou au contraire une succession de mission simple. Les missions denses sont souvent constituées de nombreux concepts de programmation (boucle, condition, fonction...) pour avoir directement un objectif concret alors que les missions simples essayent de bien séparé l'apprentissage de ceux-ci pour fournir une étude la plus progressive possible.\\

Dans certains pays, la programmation est associée au cours de sciences. Elle deviendra donc souvent un outil scientifique et perdra du même coup un peu de son côté ludique. Néanmoins, grâce à cela un plus grand nombre d'enfants pourront toucher à la programmation.\\

Dans tout les cas, il faudra trouver une manière ludique d'apprendre la programmation aux enfants. En effet, les enfants ont besoin de s'approprier la matière. Quoi de mieux qu’un jeu pour les garder attentifs à ce qu'ils réalisent.

\subsubsection{Environement de travail}
\label{paire}
L'organisation d'une leçon peut varier dans la gestion du groupe :
\begin{description}
  \item[En groupe] L'apprentissage de groupe est souvent réalisé avec un tableau interactif ou un projecteur \footnote{\url{http://scratched.media.mit.edu/resources/scratch-\%C3\%A0-la-maternelle}} ;
  \item[Par binôme] La programmation en binôme est la plus répandue dans les cours de programmation. Faire travailler les enfants par deux, les obliges à interagir entre eux et à essayer de résoudre eux même les problèmes qu'ils rencontrent. cela permet aussi de diminuer le nombre d'ordinateurs utilisés ; 
  \item[Individuel] Cette technique est très peu utilisé, car elle demande beaucoup de moyens matériels et personnels ;
  \item[A domicile] De nombreuses plates formes proposent des cours en lignes qui peuvent être suivi au rythme de chacun. Néanmoins, pour avoir de l'aide, il faudra se tourner vers une personne compétente ou vers les forums. Ces formations sont souvent plus théoriques ou demande plus de lecture, ce qui empêche les jeunes enfants de les suivre.
\end{description}

\subsection{Par qui ?}
Les personnes impliquées peuvent être de différents types suivant les organisations. Ceux-ci peuvent être des enseignants formés ou non, des passionnés ou encore des professionnels. Au-delà des personnes, il y a également le type de cours et leur création qui seront au centre de cette partie.

\subsubsection{Type d'organisation}
Différentes organisations proposent des cours de programmations.On peut déjà différencier le fait que l'informatique fasse partie du programme officiel ou pas ou plus simplement que l'apprentissage se déroule à durant les cours ou en parascolaire.\\

Dans le premier cas, les étudiants ne viennent pas tous par choix contrairement au second. Ceci peut entraîner des blocages et des réticences à participer correctement à l'activité, comme cela peut arriver pour tous les autres cours dans une école.

Pour les cours donnés en parascolaire, aussi bien de petites structures très locales, se partagent le terrain avec de grandes organisations à visée internationale. Les problèmes typiques de ce genre d'organisation sont le coût des activités et les horaires.

\subsubsection{Enseignants}
La plupart des cours sont donnés par des informaticiens ou au minimum des personnes formées à l'informatique. C'est bien entendu la manière la plus logique de travailler pour fournir un enseignement de qualité. Mais comme le nombre d'informaticiens est réduit, certaines organisations proposent des cours clés en main pour des enseignants sans connaissances particulières en programmation. Ces professeurs apprendront en même temps que leurs étudiants et seront la uniquement pour les aider à réfléchir.\\

Il n'y a pas de tendance globale sur le type de rémunération des enseignants. Ils recevront parfois des primes en fonction de la réussite de leur classe, d'autres seront payés directement par les étudiants ou par l'état, d'autres encore seront bénévoles.

\subsubsection{Création des cours}
Suivant l'organisation, les cours seront plus ou moins figés. Certaines organisations permettent à leur communauté de modifier, créer des cours tandis que d'autres ont des équipes dédiées à cette tâche.


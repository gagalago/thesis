\chapter{Validation}
\section{implémentation}
expliquer comment on a fait pour implémenter. Quel était les moyen mis en place pour faire un bon logiciel.

schéma uml

comcomber

rspec

factories

metrics (travis, gemnasium, codeclimate, coveralls, railsbestpractice)

\section{Expériences}
Cette section aborde les expérience réalisé pour ce travail. Ceci commence par une après midi chez kidscode, vient ensuite le printemps des sciences pendant lequel environ 80 élèves ont tester l'application. Enfin, vient une analyse des experience a travers des formulaire remplis par les élèves et les professeurs.
\subsection{kidscode}
\label{kidscode}
Dans le cadre de ce mémoire plusieurs expérience ont été menée. La première s'est déroulé chez Kidscode. Cette organisation a été présenté dans la section \ref{init-kidscode}. C'est une petit initiative locale qui apprend la programmation à 10 enfants âgé de 10 a 14 ans.

\subsubsection{Contexte}
\label{context-kidscode}
Il est important de définir le contexte et le publique de l'expérience car il peut avoir une influence considérable sur l'analyse des retours.\\

Lors de cette expérience le publique était un publique de 10 enfant de 10 à 14 ans ayant déjà fait une demi année de programmation dans le cadre du projet Kidscode. Le niveau de c'est enfant n'est donc pas à négligé. Ils étaient déjà habituer a compiler un programme et maîtrisaient les principaux concepts de la programmation.\\

Le but poursuivit dans cette expérience est de validé l'utilisabilité de la plateforme et également le niveau de difficulté des missions proposées. Un autre but était également de se familiariser avec le public et en petit groupe pour pouvoir pour la seconde expérience gérer un groupe plus conséquent et moins discipliner.
Ce dernier point pourrait semblé ne pas être justifier dans le cadre d'un mémoire universitaire, mais la manière d'aborder les enfants et probablement aussi important de la qualité de l'application. Pour limiter le plus possible les biais dû a un manque de pédagogie ou d'expérience dans la gestion d'un groupe d'enfant, il était important de pouvoir avoir cette première expérience.
\subsubsection{Déroulement}
L'expérience s'est déroulée pendant un peu plus d'une heure. Les enfants avaient les premières heures l'activité kidscode normalement et la dernière heure était dédier à l'expérience. 

Comme le changement d'activité était clairement marqué, le début de l'expérience fut marqué par beaucoup de perte d'attention et de dissipation. Un fois le groupe repris en main, les jeunes ont directement montré un intérêt très marqué. D'une part pour l'interface qu'ils découvraient et qui semblait leur plaire. D'autre part pour une compétition qui s'est très vite mise en place au seins du groupe.\\

Comme expliqué dans la partie précédente \ref{context-kidscode}, le public avait déjà de bonne notion de programmation par rapport au publique visé par ce mémoire. Les première missions sont donc passées très vite. Les seuls points de ralentissement étaient principalement des problèmes de lecture des consignes. 

A ce moment, les consignes étaient toutes écrites. Cela a posé un problème car les jeunes ne prenaient pas le temps de les lire et donc ne savaient pas quoi faire une fois dans les missions. Ceci malgré la possibilité de retrouver les consigne via les menu de l'application. Ce manque d'intérêt pour les consignes ne venait pas d'un problème de niveau de lecture mais simplement d'une impatience face a l'expérience et donc au manque de rigueur dû à leur âge.\\

La compétition que les jeunes ont mis en place dès le début a eu un effet bénéfique pour leur évolution car elle leur donnaient la motivation de réaliser les défit proposé. Ceci était fort marqué à chaque passage de mission. Chaque fois qu'un groupe finissait une mission il était important pour eux de le signalé et cela leur donnait une satisfaction qui les poussait réaliser la mission suivante.\\ %d'ou le fait que plein de petite mission est approprier.

Dans les délais imparti, tous sont arrivé à la mission final du chien et du chat \ref{chien-chat}, mais personne n'a réussi à la finir. Quand a sonné la fin de l'expérience, beaucoup nous ont demander comment faire pour montrer ce qu'ils avaient réaliser à leur famille et comment faire pour continuer leurs réalisations.

\subsubsection{Amélioration}

comment ca c'est déroulé. Qui y a participé. en quoi ce public est pertinent. quel était les objectifs.

\subsection{Printemps des sciences}
Cette expérience s'est dérouler dans le cadre de la semaine science infuse. Lors de cette dernière des écoles, du primaire et du secondaire, viennent participer à des animations dans les universités. C'est lors de cette semaine science infuse que l'expérience s'est passée avec quatre groupes d'enfants de différent âge et horizon.












comment ca c'est déroulé. Qui y a participé. en quoi ce public est pertinent. quel était les objectifs.

\subsection{analyse des résultats}
chiffre et analyse de notre formulaire prof et student.
















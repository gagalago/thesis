\section{Motivation}
\label{intro-motivation}
L'éducation des jeunes à la programmation peut être bénéfique à la société sur de nombreux points comme la structuration logique de l'esprit mais aussi une meilleur compréhension de l'environnement et un meilleur départ des enfants si ils souhaite partir dans l'informatique.

Ce mémoire a été sélectionner pour son coté concret dans sa finalité. Pour l'apport qu'il pourrait apporter et son impact direct qu'il pourrait avoir sur l'enseignement.
Un autre point imoprtant est l'interaction avec des enfants et le coté pédagogique.\\

L'Europe nous indique que la programmation est exellent pour la structuration logique de l'esprit apporté par la programmation. En effet la programmation demande beaucoup de rigueur, une erreur ou un oubli dans la structuration ou dans la syntaxe entraine directement une erreur. L'algorithmie qui peut être vu comme un enchainement de commande ou d'instruction est exactement l'exercice d'un raisonnement logique.

Dans les cours de la communauté francaise il y a des cours de "technologie" qui ont des programmes très libre. Dans ces cours plusieurs professeurs essaye d'apprendre l'informatique aux jeunes mais par manque de supports, d'applications adéquates et probablement aussi par manque de connaissances, ils n'enseignent que trop peu la programmation et se contente de bureautique. Rajouter la programmation aux cours de sciences serait aussi une autre bonne manière d'amener un outil d'analyse de données en plus de toutes les caractéristiques précitées de la programmation.

La technologie et l'informatique sont partout, pourtant bon nombre de personnes ne comprennent pas comment tout ca fonctionne. Il serait donc interessant que les enfants ai une idée de comment fonctionnent les objets de leur quotidien.

\section{Approche}
\label{intro-approche}
Comme des plate-forme d'apprentissage de la programmation aux jeunes existe déjà nous avons mener une étude de ces dernière pour savoir si elles pouvaient correspondre à nos!!! besoin.

Deux plate-forme étaient assez aboutie pour être potentiellement utilisable. SCRATCH et SNAP, commençons par l'analyse de SCRATCH.

SCRATCH est une plate-forme de programmation par blocs. C'est une application autonome qui s'exécute sous Windows, Mac et Linux. SCRATCH est implémenté en Squeak \footnote{Implémentation libre de SmallTalk} et est distribué sous licence Creative Commons "Attribution - ShareAlike" ce qui signifie qu'il est possible de le réutiliser et de le modifier tant que la licence est conservée et que les sources sont citées. SCRATCH a également une interface web appelée scratch 2.0 mais la licence de cette dernière n'est pas encore connue et le code n'est pas encore libéré. Ceci pose un gros problème car cela fait depuis début 2013 que la version beta de l'application web est sortie et si le code n'a pas encore été libéré on peut pensé que cette application sera sous licence propriétaire. Ceci nous empêchera de réutiliser ce code pour ce travail.

SNAP est une plate-forme de programmation par blocs qui s’exécute dans un navigateur internet. Cette plate-forme est programmée en JavaScript et est sous licence AGPL. Cette licence permet de reprendre le travail et de l'étendre. Le fait que l'application soit faite pour être exécuter dans un navigateur est un point important car c'est un des objectif de ce travail. La licence est tout à fait adapté aux objectif également dans le sens ou il est possible de repartir de cette plate-forme pour avoir un travail fini de qualité supérieur à un cas ou il fallait commencer depuis rien. Le JavaScript n'est pas un langage que nous!!! maîtrisons bien mais ce dernier reste un langage largement utiliser et ayant une large communauté.
A l'utilisation il s'avère que l'interface de SNAP est moins aboutie que celle de SCRATCH, tant au point de vue des image que des couleur et de l'allure générale. Celle de SCRATCH a par exemple des couleur plus "flash" et une impression de plus de dynamisme avec des couleur tranché et claire. L'interface de SNAP est essentiellement dans les ton sombre ce qui n'incite pas au dynamisme. Ceci s'explique par la différence de moyen pour le développement de chaque produit. En effet SNAP est une application communautaire alors que SCRATCH est le fruit d'une société, il est donc évident que les moyens mis en oeuvre ne sont pas les même. Par contre le fait que SNAP soit communautaire implique qu'il y a de forte chance pour qu'il reste sous licence open source. Du coté de SCRATCH il a été question que la version web soit mise sous licence propriétaire ce qui aurait un effet catastrophique sur le projet de ce travail.


A vue de cette analyse c'est donc SNAP qui a été choisi pour être la base de ce travail. Pour rencontrer les objectifs fixé précédemment il a fallu forker l'application et la coupler a un site web pour permettre de récupéré les projet implémenté et aussi de donner les projet dans un ordre prédéfini a l'application. Ce site permettra aussi aux jeunes de voire leur progression ainsi que de reprendre et d'amélioré leur ancien projet.

Nous allons reprendre le projet libre SNAP BYOB, en faire une interface libre de tout bouton inutile pour l'apprentissage en milieu scolaire. 

Faire tourner l'application dans un browser et sauvgarder les donnée sur un serveur distant. Grace a ce serveur faire partie gestion de classe pour facilité l'utilisation du coter du corps enseignant.

\section{Context}
\label{intro-context}
Dans un monde ou pédagogie est un sujet de société et ou l'informatique est omniprésent, on observe que peu d'interaction entre ces deux sujet. Il existe toute fois quelque recherches et initiatives. Dans les initiatives il faut encore distinguer celles qui concerne l'informatique dans son ensemble et celle qui se porte sur la programmation.

Dans ce domaine il existe encore deux écoles. Une qui prône l'apprentissage de la programmation directement avec des langage comme Python. D'autre préfère des langages simplifier comme par exemple la programmation par blocs[ref]. Ces deux école se différencie essentiellement par rapport au publique qu'elles visent. Une programmation par blocs sera plus simple dans les premiers temps et donc sera plus adapté au enfants plus jeune n'ayant aucune expérience avec la programmation. Une approche avec un langage complet comme python sera plus adapté pour des enfants plus âgé ou ayant déjà une experience. Cette dernière approche souffrira moins des restriction du langage du fait qu'il n'est pas aussi restrictif qu'un langage par bloc pour débutant.

Dans le domaine de la recherche, nous retiendrons essenciellement une publication de la Communauté Européen qui stipule que l'apprentissage de la programmation apporte un soutien a la structuration de raisonnement logique. blabalbla

Nous cloturerons cette partie avec un constat, dans une société ou l'informatique occupe une place prédominante, nous constatons que la connaissance de la programmation n'est pas beaucoup valoriser et stimulé 





On va parler des programmes pour apprendre à des jeunes la programmation par blocs.

Notre monde est de plus en plus tourné vers l'informatique mais peu savent programmer blablabla.

\section{Contexte}
\label{intro-context}
Dans un monde où la pédagogie est un sujet de société et où l'informatique est omniprésente, on observe peu d'interactions entre ces deux sujets. Il existe toutes foi quelques recherches et initiatives à ce sujet. Il faut encore distinguer celles qui concernent l'informatique dans son ensemble et celles qui se portent sur la programmation.\\

Il existe deux écoles d'apprentissage de la programmation. La première prône l'apprentissage de la programmation directement avec des langages comme Python. D'autres préfèrent des langages simplifiés par exemple la programmation par blocs. 

Ces deux écoles se différencient essentiellement par rapport au public qu'elles visent. Une programmation par blocs sera plus simple dans le premier temps et donc sera plus adaptée aux enfants plus jeunes n'ayant aucune expérience avec la programmation. Une approche avec un langage complet comme python sera plus adaptée pour des enfants plus âgés ou ayant déjà une expérience. Cette dernière approche souffrira moins des restrictions du langage du fait qu'il n'est pas aussi restrictif qu'un langage par bloc pour débutant.\\

Dans le domaine de la recherche, nous retiendrons essentiellement une publication de la communauté européenne qui stipule que l'apprentissage de la programmation apporte un soutien a la structuration de raisonnement logique.\\

Nous clôturerons cette partie avec un constat, dans une société où l'informatique occupe une place prédominante, nous constatons que la connaissance de la programmation n'est pas encore beaucoup valorisée et stimulée. 

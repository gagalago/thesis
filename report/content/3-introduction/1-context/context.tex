\section{Context}
\label{intro-context}
Dans un monde où la pédagogie est un sujet de société et ou l'informatique est omniprésent, on observe peu d'interaction entre ces deux sujet. Il existe toute fois quelques recherches et initiatives. Il faut encore distinguer celles qui concerne l'informatique dans son ensemble et celle qui se porte sur la programmation.\\

Il existe deux écoles d'apprentissage de la programmation. Une qui prône l'apprentissage de la programmation directement avec des langage comme Python. D'autre préfère des langages simplifiés comme par exemple la programmation par blocs. 

Ces deux école se différencient essentiellement par rapport au public qu'elles visent. Une programmation par blocs sera plus simple dans le premier temps et donc sera plus adaptée aux enfants plus jeunes n'ayant aucune expérience avec la programmation. Une approche avec un langage complet comme python sera plus adapté pour des enfants plus âgé ou ayant déjà une experience. Cette dernière approche souffrira moins des restriction du langage du fait qu'il n'est pas aussi restrictif qu'un langage par bloc pour débutant.\\

Dans le domaine de la recherche, nous retiendrons essenciellement une publication de la Communauté Européen qui stipule que l'apprentissage de la programmation apporte un soutien a la structuration de raisonnement logique.\\

Nous cloturerons cette partie avec un constat, dans une société ou l'informatique occupe une place prédominante, nous constatons que la connaissance de la programmation n'est pas encore beaucoup valorisée et stimulée. 
